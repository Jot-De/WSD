%%%%%%%%%%%%%%%%%%%%%%%%%%%%%%%%%%%%%%%%%%%%%%%%%%%%%%%
%% Engineer & Master Thesis, LaTeX Template          %%
%% Copyleft by Piotr Woźniak & Artur M. Brodzki      %%
%% Faculty of Electronics and Information Technology %%
%% Warsaw University of Technology, Warsaw, 2019     %%
%%%%%%%%%%%%%%%%%%%%%%%%%%%%%%%%%%%%%%%%%%%%%%%%%%%%%%%

\documentclass[
    left=2.5cm,         % Sadly, generic margin parameter
    right=2.5cm,        % doesnt't work, as it is
    top=2.5cm,          % superseded by more specific
    bottom=3cm,         % left...bottom parameters.
    bindingoffset=6mm,  % Optional binding offset.
    nohyphenation=false % You may turn off hyphenation, if don't like. 
]{eiti/eiti-thesis}

\usepackage[polish]{babel}
\usepackage[
    backend=bibtex,
    style=ieee
]{biblatex}
\usepackage{csquotes}

\graphicspath{{img/}}             % Katalog z obrazkami.
\addbibresource{bibliografia.bib} % Plik .bib z bibliografią

%----------------------------------------
% Twierdzenia i definicje;
% tutaj ew. tłumaczymy te terminy
% na inne języki
%----------------------------------------
\newtheorem{theorem}{Twierdzenie}
\newtheorem{lemma}{Lemat}
\newtheorem{corollary}{Wniosek}
\newtheorem{definition}{Definicja}
\newtheorem{axiom}{Aksjomat}
\newtheorem{assumption}{Założenie}

%----------------------------------------
% Spis rysunków, tablic i załączników;
% tutaj ew. tłumaczymy te terminy
% na inne języki
%----------------------------------------
\AtBeginDocument{
    \renewcommand{\listfigurename}{Spis rysunków}
    \renewcommand{\listtablename}{Spis tabel}
    \renewcommand{\tablename}{Tabela}
}

\begin{document}

%--------------------------------------
% Strona tytułowa
%--------------------------------------
\MasterThesis % dla pracy inżynierskiej mamy \EngineerThesis
\title{
    SmartPark
    Część A: Identyfikacja problemu.
}
\engtitle{ % Tytuł po angielsku do angielskiego streszczenia
    Część A: Identyfikacja problemu.
}
% TODO: Dodajcie swoje indeksy plz.
\subject{Wieloagentowe Systemy Decyzyjne}
\author{Jan Dubiński | 271165}
\secondAuthor{Joanna Kaleta | 271181}
\thirdAuthor{Aleksander Ogonowski | 267381}
\fourthAuthor{Maria Oniszuczuk | 271199}
\fifthAuthor{Kornel Szymczyk | 267778}
\date{\the\year}
\maketitle

%--------------------------------------
% Spis treści
%--------------------------------------
repozytorium git:  https://github.com/Jot-De/WSD/

\hfill \break

\thispagestyle{empty}
\tableofcontents

%--------------------------------------
% Rozdziały
%--------------------------------------
\newpage
\section{Opis problemu}

Coraz większym problemem rozwijających się aglomeracji staje się nadmierne zagęszczenie ruchu samochodowego, a co się z tym wiąże - trudność w znalezieniu miejsc parkingowych. Brak łatwo dostępnej informacji o wolnych miejscach do parkowania zarówno na pobliskich parkingach, jak i w przestrzeni publicznej prowadzi do irytacji kierowców oraz marnowania ich czasu. Jedynym sposobem aby przekonać się, że wybrany parking jest w pełni obłożony, jest pojawienie się tam osobiście. Kierowcy w poszukiwaniu miejsc parkingowych krążą po ulicach dodatkowo potęgując korki uliczne. Stając w miejscach publicznych, w których parkowanie jest niedozwolone, dodatkowo utrudniają ruch pieszym i pozostałym kierowcom, jednocześnie narażając się na nieprzyjemne konsekwencje finansowe. Naszą propozycją mającą ułatwić parkowanie i zredukować ruch uliczny jest system XXX.
\newpage
\section{Słowny opis koncepcji systemu}

Proponowanym przez nas rozwiązaniem problemu omówionego w punkcie pierwszym jest system wieloagentowy ułatwiający znalezienie miejsca parkingowego, który pozwoli na znaczną redukcję liczby krążących w jego poszukiwaniu samochodów.

System będzie składać się z sieci parkingów oraz aplikacji mobilnej dla kierowców pojazdów. Przy wykorzystaniu aplikacji, użytkownik będzie mógł zrealizować wyszukanie oraz nawigację do wybranego pobliskiego parkingu po wcześniejszym zgłoszeniu swojej lokalizacji.

System po zgłoszeniu chęci zaparkowania proponuje pięć pobliskich parkingów z wolnymi miejscami, spośród których kierowca może wybrać ten do którego chce się udać. Kluczowe w systemie jest pozwolenie kierowcy na znalezienie jak najbliższego odpowiadającego mu parkingu z przynajmniej jednym wolnym miejscem. Celem parkingu jest natomiast pozbycie się pustych miejsc, czyli jego zapełnienie.

Każde miejsce parkingowe powinno zostać wyposażone w czujniki ultradźwiękowe wykrywające zajętość miejsca oraz kamerę identyfikującą konkretny pojazd. W przypadku zwolnienia miejsca powinna podnosić się blokada uniemożliwiająca wjazd niezgłoszonemu  na dany parking samochodowi. Każdy parking agreguje konkretną liczbę miejsc postojowych na przykład na odcinku konkretnej ulicy i posiada informacje o ich zajętości. Każde miejsce parkingowe po zwolnieniu wysyła adekwatną informację do parkingu, do którego przynależy, dzięki czemu może on zgłosić chęć przyjęcia następnego kierowcy. Gdy kierowca zadeklaruje chęć zaparkowania, po czym jest nawigowany do określonego parkingu, powoduje to zwiększenie się jego zajętości. Parking oczekuje na kierowcę przez pewien ustalony czas, nie przyjmując kolejnych deklarujących chęć pojazdów, jeżeli nie ma w swoim obrębie wolnych miejsc. Dzięki takiemu rozwiązaniu mamy pewność, że po dojechaniu na parking znajdziemy puste miejsce. Gdy kierowca podjedzie do miejsca na przydzielonym parkingu i zostanie poprawnie zidentyfikowany przez kamerę poprzez numer rejestracyjny, blokada opuszcza się, kierowca parkuje na miejscu postojowym, a miejsce informuje parking o jego prawidłowym zajęciu, zwiększając zajętość jego aż do chwili opuszczenia miejsca przez pojazd. Przydzielony parking docelowo powinien być oznaczony np. przy wykorzystaniu Open Street Map jako region do którego prowadzony jest kierowca. 

\newpage
\noindent Podczas korzystania z aplikacji można wydzielić kilka  typowych scenariuszy:


\noindent Scenariusz 1 - główny - wyszukanie miejsca parkingowego: \\
\hspace*{1cm} 1. Użytkownik deklaruje w aplikacji chęć znalezienia miejsca parkingowego.  \\
\hspace*{1cm} 2. System wyszukuje parkingi z przynajmniej jednym wolnym miejscem w okolicy użytkownika komunikując się z pobliskimi parkingami. \\
\hspace*{1cm} 3. System proponuje parkingi, które użytkownik może zaakceptować.  \\
\hspace*{1cm} 4. Użytkownik wybiera w aplikacji jeden proponowanych przez system parking. \\
\hspace*{1cm} 5. System nawiguje kierowcę do parkingu. \\
\hspace*{1cm} 6. Kierowca stawia się na dowolnym miejscu parkingowym w obrębie parkingu \\

\noindent Scenariusz 2 - użytkownik oddala się od parkingu \\
\hspace*{1cm} 1-5. Jak w scenariuszu głównym \\
\hspace*{1cm} 6. Kierowca nie kieruje się na parking (jego pozycja oddala się). \\
\hspace*{1cm} 7.Parking sprawdza czy proces postępuje \\
\hspace*{1cm} 8.Kierowca nadal się oddala  \\
\hspace*{1cm} 9. Parking dokonuje zwolnienia miejsca parkingowego \\

\noindent Scenariusz 3 - alternatywny do scenariusza 1 - użytkownik odrzuca proponowane parking \\
\hspace*{1cm} 1-4. 	Jak w scenariuszu 2 \\
\hspace*{1cm} 5.	Użytkownik nie wybiera zasugerowanego przez system parkingu \\
\hspace*{1cm} 6. Kierowca oddala się o pewną odległość \\
\hspace*{1cm} 7. Powrót do punktu 1  scenariusza głównego \\

\noindent Scenariusz 4 - użytkownik rezygnuje z rezerwacji \\
\hspace*{1cm} 1-5. Jak w scenariuszu głównym \\
\hspace*{1cm} 6. Kierowca rezygnuje z miejsca parkingowego \\
\hspace*{1cm} 7. Parking dokonuje zwolnienia miejsca parkingowego \\



\noindent Scenariusz 5 - brak wolnych miejsc parkingowych \\
\hspace*{1cm} 1-3. 	Jak w scenariuszu głównym \\
\hspace*{1cm} 4.	Aplikacja wyświetla komunikat o braku wolnych miejsc parkingowych na każdym typie parkingu. \\
\hspace*{1cm} 5. Kierowca oddala się o pewną odległość \\
\hspace*{1cm} 6. Powrót do punktu 1 scenariusza głównego. \\

\newpage
\section{Architektura systemu}
Celem nadrzędnym systemu jest maksymalne zapełnienie parkingów.

System opiera się na dwóch rodzajach agentów
\begin{itemize}
\item kierowców samochodów z zainstalowaną aplikacją
\item parkingów agregujących miejsca parkingowe
\end{itemize}

Kierowcy dążą do znalezienie parkingu jak najbliżej aktualnego miejsca.
Parkingi dążą do maksymalnego zapełnienie się.
System wysyła kierowcy propozycję miejsca parkingowego na parkingu, który znajduje się najbliżej jego aktualnego położenia i który posiada wolne miejsca. Po zaakceptowaniu przesłanej propozycji następuje zarezerwowanie miejsca postojowego dla klienta na podstawie jego unikalnego identyfikatora.\\

System wymaga, aby samochód cyklicznie wysyłał zapytania do parkingów otrzymując w odpowiedzi ich położenie.

System uzgadniania miejsc nie może dopuścić do sytuacji gdy przy jednoczesnym zgłoszeniu chęci parkowania przez dwa lub więcej samochody zostaną przydzielone dwa samochody na jedno wolne miejsce parkingowe.

arking nadzoruje, czy kierowca nie oddala się od niego po zgłoszeniu chęci zaparkowania przez dłuższy czas lub nie wysłał zgłoszenia dotyczącego anulowania rezerwacji. W przypadku wystąpienia jednego z tych zdarzeń parking odwołuje rezerwację zwalniając przydzielone dla samochodu miejsce.}

Miejsca parkingowe są natomiast aktorami będącymi nieaktywnymi przez większość czasu. Ich zadanie polega jedynie na wysyłaniu komunikatów do parkingu, do którego przynależą. Jest to komunikat o zajęciu i zwolnieniu miejsca przez kierowcę. Na tej podstawie parking wie ile samochodów znajduje się na nim, a ile może przyjąć. Od liczby samochodów, które parking może przyjąć należy odjąć również samochody, które zgłosiły chęć parkowania, ale jeszcze nie dojechały. Nigdy jednak nie zostaje przyjęte więcej samochodów niż liczba wolnych miejsc.

Kierowcy komunikują się z parkingami zgłaszając chęć parkowania lub odrzucając sugerowany parking. Generalnie nie ma potrzeby, aby samochody prowadziły komunikację między sobą.

Naszymi propozycjami rozwoju systemu w kolejnych etapach jest:
\begin{itemize}
\item dodanie nadzoru nad czujnikami - sygnalizowanie awarii i prowadzenie napraw
\item udoskonalenie algorytmu monitorującego zachowanie kierowcy tak, aby uwzględniał on korki, objazdy, wypadki itd.
\end{itemize}


\begin{figure}[H]
    \label{fig:architektura}
    \centering \includegraphics[width=1.1\linewidth]{archi.png}
    \caption{Architektura systemu.}
\end{figure}

\clearpage
\newpage
%--------------------------------------------
% Literatura
%--------------------------------------------


\printbibliography

%--------------------------------------------
% Spisy (opcjonalne)
%--------------------------------------------
\newpage

% Wykaz symboli i skrótów.
% Pamiętaj, żeby posortować symbole alfabetycznie
% we własnym zakresie. Ponieważ mało kto używa takiego wykazu, 
% uznałem, że robienie automatycznie sortowanej listy
% na poziomie LaTeXa to za duży overkill. 
% Jest tylko proste i oczywiste makro \acronym, 
% które dodaje postawowe formatowanie. 
% //AB
% \vspace{0.8cm}
% \section*{Wykaz symboli i skrótów}
% \acronym{EiTI}{Wydział Elektroniki i Technik Informacyjnych}
% \acronym{PW}{Politechnika Warszawska}

\listoffigures              % Spis obrazków. 
\vspace{1cm}                % vertical space
% \listoftables               % Spis tabel. 
% \vspace{1cm}                % vertical space


% \listofappendices           % Spis załączników
% Załączniki
% \appendix

% \newpage
% \newappendix{Nazwa załącznika 1}
% \lipsum[1]

% \newpage
% \newappendix{Nazwa załącznika 2}
% \lipsum[1]

% Używając powyższych spisów jako szablonu,
% możesz tu dodać swój własny wykaz bądź listę, 
% np. spis algorytmów. 

\end{document} % Dobranoc. 

